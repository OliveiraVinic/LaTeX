\documentclass{provas}
\usepackage{graphicx}
\usepackage{pstricks,pst-text}
\usepackage[utf8x]{inputenc}
\usepackage[brazil]{babel}
\usepackage{times}
\usepackage[T1]{fontenc}
\usepackage{amsmath}     
\usepackage{ifpdf}
\usepackage{amssymb}
\usepackage{wasysym}

\usepackage{cancel}

\let\svthefootnote\thefootnote

\curso{Curso}
\disciplina{Disciplina}
\data{Data da Prova}

\begin{document}

\pagestyle{empty}

\cabecagab

\vspace{0.5cm}
\begin{center}
 \textbf{\underline{Prova Número -- Gabarito}}
\end{center}

\vspace{0.5cm}
%exemplo de questão de gabarito (sem enunciado, porém com pontuação máxima)
\questaoG{Nota}

Digite a resolução da questão, com comentários ajuda bastante

\begin{center}
 \rule{0.5\textwidth}{0.02cm}
\end{center}
%%%%%%%%%%%%%%%%%%%%%%%%%%%%%%%%%%%%%%%%%%%%%%%%%%%%%%%%%%%%%%%%%%%%%%%%%%%%%%%%%%%%%%%%%%%%%%%%%%%%%%

%exemplo de questão de gabarito (sem enunciado, porém com pontuação máxima)
\questaoG{Nota}

Digite a resolução da questão, com comentários ajuda bastante

(sim está repetido, apenas para mostrar a estrutura)

\begin{center}
 \rule{0.5\textwidth}{0.02cm}
\end{center}
%%%%%%%%%%%%%%%%%%%%%%%%%%%%%%%%%%%%%%%%%%%%%%%%%%%%%%%%%%%%%%%%%%%%%%%%%%%%%%%%%%%%%%%%%%%%%%%%%%%%%%


\end{document}







