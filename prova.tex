\documentclass{provas}
\usepackage{graphicx}
\usepackage{pstricks,pst-text}
\usepackage[utf8x]{inputenc}
\usepackage[brazil]{babel}
\usepackage{times}
\usepackage[T1]{fontenc}
\usepackage{amsmath}     
\usepackage{ifpdf}
\usepackage{amssymb}


\let\svthefootnote\thefootnote

\curso{Curso}
\disciplina{Disciplina}
\data{Data da Prova}
\nome{branco} %colocar o nome do aluno para personalizar a prova, se escrito branco deixará uma linha em branco

\begin{document}

%explicação geral sobre a prova
\let\thefootnote\relax\footnote{\textit{\textbf{Atenção:} a consulta é permitida apenas para material manuscrito.}}
\pagestyle{empty}

\cabeca

\vspace{0.5cm}
\begin{center}
 \textbf{\underline{Prova Número}}
\end{center}

%informações gerais da prova
\vspace{0.5cm}
\texto{\textit{Informações:} \textit{geralmente eu coloco aqui as constantes para a prova}.}

%exemplo de questão padrão
\vspace{0.75cm}
\questao{Nota}{Escreva aqui o enunciado e a pergunta da questão.}

%exemplo de questão com itens
\vspace{0.75cm}
\questaoI{Escreva aqui o enunciado da questão

\textit{(Nota)} \textbf{a)} Primeiro item.

\textit{(Nota)} \textbf{b)} Segundo item.}

%exemplo de questão padrão
\vspace{0.75cm}
\questao{Nota}{Escreva aqui o enunciado e a pergunta da questão. (só para mostrar que a numeração é automática)}

%exemplo de questão única
\vspace{0.75cm}
\questaoU{Muito similar a questão com itens, porém não recebe numeração.}


\end{document}







